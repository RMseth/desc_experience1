\documentclass[10pt,a4paper]{report}
\usepackage[latin1]{inputenc}
\usepackage{amsmath}
\usepackage{amsfonts}
\usepackage{amssymb}
\usepackage{graphicx}
\author{Seth Bertlshofer\\Alexis Tyler\\Dustin Ginos\\Kevin Burgon}
\title{Desc. Experience 1}
\graphicspath{{img/}}
\begin{document}
	\maketitle
	\textbf{Sub-Experience One: Fun and Games}\\
		\begin{center}
			Figure 1.\\
			\includegraphics[scale=.5]{e1.png}
			\newline
			\newline
		\end{center}
		
	\textbf{Solution: }Player $A$ can always win if she employs a particular strategy for each move.  Let $G = (V, E)$ be a NIM graph where $x, y, z \in V$ ($V$ being the set of vertices in $G$) and $xy, yz, zx \in E$.  Assume that the graph never has the same number of edges between all vertices. The strategy that will be used is this: For each turn, player $A$ removes edges on one vertex such that $|xy| = |yz|$, $|yz| = |zx|$, and $|zx| = |xy|$, where $|E|$ is the number of edges between two vertices. Since the rule of the game is that the winner is the player who removes the last edge, we know that the winner is also the one who removes the second to last edge.  If that is the case, then the player who removes edge that breaks the cycle in the graph loses. The player that is forced to break the cycle is also the player who is forced to remove an edge such that either $|xy| = 1$, $|yz| = 1$, or $|zx| = 1$.  In order for player $A$ to force player $B$ to do that, player $A$ needs to force player $B$ to remove an edge from the pair of vertices with the minimum number of edges.  In order to do that, player $A$ needs to remove edges on every turn such that $|xy| = |yz|$, $|yz| = |zx|$, and $|zx| = |xy|$. Therefore, if the graph is such that $|xy| = |yz|$, $|yz| = |zx|$, and $|zx| = |xy|$, player $B$ is forced to remove edges until they break the cycle, and player $A$ will always win.\\
	
	%In that case, the strategy consists in leaving the graph such that player $B$ is forced to remove an edge that breaks the cycle.  In order to accomplish that the goal is also to force player $B$ to remove edges until they end a turn where $|xy| = 1$, $|yz| = 1$, or $|zx| = 1$.  In order to accomplish this player $A$ needs to avoid removing edges from the pair of vertices with the minimum number of edges.  If player $A$ never removes edges from the pair of vertices with the minimum number of edges player $B$ is forced to do so, therefore causing player $B$ to eventually remove an edge that results in a pair of vertices with only one edge between them.  In order for player $A$ to avoid  removing from the minimum number of edges, player $A$ always needs to remove edges such that $|xy| = |yz|$, $|yz| = |zx|$, and $|zx| = |xy|$.\\

	\textbf{Sub-Experience Two: The Binary Addressing Graph}
		\begin{enumerate}
			\item $|V(Q_n)| = 2^n$
			\begin{center}
				Figure 2.\\
				\includegraphics[scale=.5]{2_1.png}
			\end{center}
			Because this solution is looking for a binary power we can prove this with a Karnaugh Map (Figure 2) What this shows in the relation between the vertices.  If there is a relation then 1's will be adjacent to one another.  We can see that because the pattern is that is created by the Karnaugh map there can be no relations between vertices.
			\item $Q_n$ is an n-regular graph: that is, $deg (\vec{v}) = n $ for each $\vec{v} \epsilon V(Q_n)$.
			\item $Q_n$ is \textit{bipartite}; that is, $V(Q_n)$ consists of two sets, say X and Y such that $X\cap Y = \varnothing$ and the only edges of $Q_n$ have one end-vertex in X and the other in Y (so X and Y induce graphs with no edges).
			\item $Q_n$ is Hamiltonian for $n \geq 2$.
		\end{enumerate}
		
		
	\textbf{Sub-Experience Three: Space Station Problems}\\
	
		\begin{center}
			\includegraphics[scale=.5]{e3.png}
			\newline
			\newline
		\end{center}
	\textbf{Sub-Experience Four: Regions Determined by Chords of a Circle}\\
	
		\begin{center}
			%\includegraphics[scale=.5]{.png}
	
		\end{center}
	\textbf{Sub-Experience Five: Survivors in a Tournament.}\\
	
		\begin{center}
			\includegraphics[scale=.5]{e5.png}
			\newline
			\newline
		\end{center}
	\textbf{Sub-Experience Six: Better-Than-Good Will Hunting}\\
		Verify that the graph G (below) has a diameter of 2.
	
		By using Lemma 2 we can prove that this graph does indeed have a diameter of 2.  
		\begin{center}
			\includegraphics[scale=.5]{e6.png}
			\newline
			\newline
		\end{center}
	\textbf{[Bonus] Sub-Experience Seven: A Matter of Life and Death}\\	
	
		\begin{center}
			%\includegraphics[scale=.5]{.png}

		\end{center}
\end{document}